\chapter{\ecal simualtion study for \upgradetwo}
\label{chap:ecal}

%\section{Silicon tungsten \ecal simulation study}
\section{Introduction}
\label{sec:Introduction}

It is generally agreed within the LHCb collaboration 
and the community of flavour physics that an upgrade of the LHCb detector to make full exploitation of the High-Luminosity LHC (HLLHC), 
which provides an opportunity to investigate the open questions in flavour physics\supercite{LHCb-PII-EoI,LHCb-PII-Physics,Strategy:2019vxc}. 
This proposed upgrade is referred to as the LHCb \upgradetwo, 
which should be capable of taking data at an instananeous luminosity of up to $1.5\times10^{34}\cm^{-2}\cdot\mathrm{sec}^{-1}$. 
The goal is to collect a data sample of an integrated luminosity of $300\invfb$ after Run 6\supercite{LHCb-PII-EoI}.
Among the biggest challenges for the Upgrade are the large number of proton-proton interactions in a single bunch crossing (pile-up) 
and the large amount of radiation that the detector has to tolerate. 
%A plethora of efforts have been put by the LHCb collaboration since several years ago.

The present \ecal has good performance with run1 and run2 luminosity. 
There is not a certaiy answer if it can handle a very high luminosity, 
it is necessary to take a look at the performance of current \ecal with high luminosity in the parameterized simulation first. 
From the parameterized simulaiton,
the possible methods are studied to optimize the detector and let it conquer the pile-up effect.
There are many factors that could affect the ECAL performance, 
like intrinsic energy resolution and so on.
By improving these factors, 
the performance of \ecal might be promoted.
Besides,
it is commonly agreed that high precision time information is needed to deal with the pile-up,
which is the major item studied in the simulation.
And high granularity is another requirement for the Upgrade ECAL, 
which can greatly help reducing the occupancy due to the high luminosity.

%We try to   which one has the largest impact. 
%So we can study the effect when changing the geometry and different shower shape when we use a different material.

For the calorimeter, 
the main idea is that it should allow for the reconstruction and analyses of the decay modes involving neutral particles.
This chapter will discuss the requirements to \ecal for \upgradetwo in the parameterized simulation first.
A detailed silicion-tungsten (SiW) \ecal is discussed then, 
\ie a sampling ECAL with silicon as the active material and tungsten as the absorbers.

%The CALICE collaboration has been making the R\&D of the techniques of SiW ECAL has been extensively studied by the CALICE collaboration since almost twenty years ago.
%Its first design and electronic commissioning of the physics prototype was published in 2008\supercite{Anduze:2008hq}, 
%and the recent results are the study of the effects of hadronic shower in the SiW ECAL physics prototype\supercite{Eigen:2019ccp}. 
%This technique was adopted by CMS for the CMS phase-2 upgrade endcap calorimeter called HGCAL\supercite{CERN-LHCC-2017-023}. 
%A lot of R\&D activities are carried out because the HLLHC introduces challenges in radiation resistance and good time resolution. 
%The first beam tests of the prototype silicon modules for the CMS HGCAL show promising results\supercite{Akchurin:2018rpm}. 
%This indicates that a similar technique could be used for the LHCb detector, which detect particles in the forward region. 
%The first simulation results of the SiW ECAL for the LHCb Upgrade II are shown in this note.






