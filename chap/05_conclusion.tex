
\section{Conclusion}
\label{sec:conclusion}
With the $3.0\invfb$ data collected in 2011 and 2012, an observation \LbLckkpi decay channel has been performed.  Taking \LbLcDs as the normalization channel, the ratio of products of branching fraction and fragmentation fraction is measured to be
\begin{equation}
\frac{\BR( \LbLckkpi)}{\BR( \LbLcDskkpi)}=  1.71 \pm 0.069 \stat \pm 0.077 \syst.
\end{equation}

Including the branching fraction of $\BR(\Dskkpi)=(5.45\pm0.17)\%$ from PDG, we can obtain the ratio of branching fraction between \LbLckkpi and \LbLcDs:
\begin{equation}
	\frac{\BR( \LbLckkpi)}{\BR( \LbLcDs)}=  (9.33 \pm 0.38\stat \pm 0.42 \syst \pm 0.29 (\BR))\times10^{-2},
\end{equation}
where the uncertainties are statistical, systematic and systematic due to uncertainty of  $\BR(\Dskkpi)$.

The predicted pentaquark can be checked in the $\Lc\Kp$ mass spectrum\supercite{Huang:2004tn}. 
Figure.~\ref{fig.Resonance_2body} shows the invariant mass distribution of $\Lc\pi^-$, obtained with the background subtracted data. 
There is no significant peak in this distribution. 
Besides, 
there is no clear sign of $\Xic$ from the invariant mass distribution of $\Lc\Km$.




