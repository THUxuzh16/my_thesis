\section{Amplitude analysis}

Then, a full amplitude analysis is performed to examine the contributions of the two $\ZP$ states and possible $\ZC$ state.
Isobar model approach is used in which each resonance is described by a Breit-Wigner.
%or a coupled-channel Flatt{\'e} amplitude.~\footnote{
%There are also three successful models for baryon amplitudes: MAID (http://portal.kph.uni-mainz.de/MAID//), SAID (http://gwdac.phys.gwu.edu/) and the Bonn-Gatchina model (http://arxiv.org/pdf/0911.5277v2.pdf).}
The amplitude formalism has been described in the $\Bp\to\jpsi\proton\Kp$ analysis above, Section~\ref{sec:03_matrixelement}, 
by replacing variables with the particle $\Kp$ by these of the $\pim$,
the $\proton$ by these of the $\phi$,
and $\Bp$ by these of the $\Lb$.
There are 6 fitting observables: invariant mass of $p\pi$, $m(\proton\pi)$, 
and five decay angles denoted as $\POmega$, 
including
(i) cosine of helicity angle of the $\Lb$ decays $\cos\,\theta_{\Lb}$;
(ii) cosine of helicity angle of the $N^*$ decays $\cos\,\theta_{N^*}$;
(iii) cosine of helicity angle of the $\jpsi$ decays $\cos\,\theta_{\jpsi}$;
(iv) angle between $\Lb$ and $N^*$ decay planes $\phi_{\pi}$; and
(iv) angle between $\Lb$ and $\jpsi$ decay planes $\phi_{\mu}$.

%To examine such possibility, we need to include the $\ZC^-$ decay chain in the fit.  
%The matrix element for such chain is derived as

%\begin{align}
%\Mat_{\lambda_{\Lb},\,\lambda_p^{\ZC},\,\Delta\lambda_\mu^{\ZC}}^{\,\,\ZC} \equiv
%&
%\sum\limits_{f}\sum\limits_{\lambda_{\ZC}}\sum\limits_{\lambda_\psi^{\ZC}}
% \H^{\Lb\to {\ZC}_f p}_{\lambda_{\ZC},\,\lambda_{p}^{\ZC}} \,\,
% D^{\,\,\frac{1}{2}}_{\lambda_{\Lb},\,\lambda_{\ZC}-\lambda_p^{\ZC}}(\phi_{\ZC},\theta_{\Lb}^{\ZC},0)^*\,\,
% \notag \\
%& ~ \H^{{\ZC}_f\to\psi \pi}_{\lambda_{\psi}^{\ZC}}\,\,
% D^{\,\,J_{{\ZC}_f}}_{\lambda_{\ZC},\,\lambda_{\psi}^{\ZC}}(\phi_\psi^{\ZC},\theta_{\ZC},0)^*\,
% R_{{\ZC}_f}(M_{\psi \pi}) \,
%D^{\,\,1}_{\lambda_\psi^{\ZC},\,\Delta\lambda_\mu}(\phi_{\mu}^{\ZC},\theta_\psi^{\ZC},0)^* \,\,
%\label{eq:Zc_matrixelement_partial}
%\end{align}
%where the angles and helicity states carry the superscript or subscript $\ZC$ to distinguish them from those defined for the other decay chains. The sum over $f$ allows for the possibility of contributions from more than one $\ZC^-$ resonance. There are 2 (4) helicity couplings $\H^{\Lb\to {\ZC}_f p}_{\lambda_{\ZC},\,\lambda_{p}^{\ZC}}$ for $J_{{\ZC}_f}$ = $0 (\ge1)$ to determined from the data. %When using $LS$ couplings for $J_{{\ZC}_f}\ge1$ state, the former couplings can be reduced to only 1 (3) free couplings if only the lowest (the lowest two) values of $L$ in the $\Lb$ decays are considered.
%For some spin and parity of $\ZC$ state, such as $1^+$, there is an additional ratio of two independent $LS$ couplings for the ${\ZC}_f\to\psi \pi$ decay, $B_{L_{\rm min}+2,\,1}^{{\ZC}_f\to\psi \pi}/_{L_{\rm min},\,1}^{{\ZC}_f\to\psi \pi}$, to be determined, where $L_{\rm min}$ denotes the allowed lowest $L_{{\ZC}_f}$ in the ${\ZC}_f$ decay and there is a second allowed $L_{{\ZC}_f}=L_{\rm min}+2$.%; it can be set to zero if neglecting the contribution of the highest value of $L_{{\ZC}_f}$ in the $\ZC^-$ decay.
%We allow all the couplings to vary in the fit.
%The angle calculations are analogous to these defined in the $\ZP$ decay chain by only interchange $p$ and $\pi^-$.


Except for the exotic baryonic $\ZP$ states, 
exotic mesonic states $\ZC^-\to\jpsi \pim$ could also contribute to this decay. 
The matrix elements for the $\jpsi N^*$, $P_c^+\pim$ and $\jpsi\pim$ decay chains, 
are similar to that in the previous $\Lb\to\jpsi\proton\pim$ study~\supercite{LHCb-PAPER-2016-015}.
Furthermore, the proton and muon helicity states in each of the exotic decay chains must be expressed in the basis of helicities in the $N^*$ decay chain, 
which is incorporated in the total decay matrix element as follows
\begin{align}
\left| \Mat \right|^2 =
\sum\limits_{\lambda_{\Lb}}
\sum\limits_{\lambda_{p}}
\sum\limits_{\Delta\lambda_{\mu}}
&\left|
\Mat_{\lambda_{\Lb},\,\lambda_p,\,\Delta\lambda_\mu}^{N^*}
+
e^{i\,{\Delta\lambda_\mu}\alpha_{\mu}}\,
\sum\limits_{\lambda_p^{\ZP}}
d^{\,\,\frac{1}{2}}_{\lambda_p^{\ZP},\,\lambda_p}(\theta_p)\,
\Mat_{\lambda_{\Lb},\,\lambda_p^{\ZP},\,\Delta\lambda_\mu}^{\ZP}  \right. \notag \\
&+ \left.
e^{i\,{\Delta\lambda_\mu}\alpha^{\ZC}_{\mu}}\,
\sum\limits_{\lambda_p^{\ZC}} e^{i\lambda_p^{\ZC}\alpha_p^{\ZC}}
d^{\,\,\frac{1}{2}}_{\lambda_p^{\ZC},\,\lambda_p}(\theta^{\ZC}_p)\,
\Mat_{\lambda_{\Lb},\,\lambda_p^{\ZC},\,\Delta\lambda_\mu}^{\ZC}
\right|^2,
\label{eq:total_matrixelement}
\end{align}
where the matrix element for each of the decay chains is denoted by the superscripts, 
and $\theta_p$ ($\theta_p^{\ZC}$) is the polar angle in the $p$ rest frame between the boost directions from the $N^*$ and $\ZP^+$ ($\ZC^-$) rest frames, 
and $\alpha_\mu$ ($\alpha_\mu^{\ZC}$) is the azimuthal angle correcting for the difference between the muon helicity states 
in the $N^*$ and $\ZP^+$ ($\ZC^-$) decay chains. 
Unlike in the $\ZP^+$ chain, 
the azimuthal angle ($\alpha_p^{\ZC}$) aligning the two proton helicity frames is not zero.







\subsection{Likelihood construction with sPlot technique}
A so-called \emph{sFit} approach is used in this analysis.
In such approach the total PDF is equal to the signal PDF, 
as the background is subtracted from the log-likelihood sum using the \sWeights.
The signal PDF is obtained by multiplying the decay matrix element $\Mat$ squared with the selection efficiency, 
$\epsilon(m_{p\pi},\POmega)$,
\begin{equation}
\label{SUPPeq:sigpdf}
\PDF_{\rm sig}(m_{p\pi},\POmega|\Pars)
=\frac{1}{I(\Pars)}\left|\Mat(m_{p\pi},\POmega|\Pars)\right|^2
\Phi(m_{p\pi})
\epsilon(m_{p\pi},\POmega),
\end{equation}
where
$\Phi(m_{p\pi})$ is the phase space function equal to $p\cdot q$, 
with $p$ being the momentum of the $p\pi$ system (\ie\ $\NStar$) in the $\Lb$ rest frame, 
and $q$ being the momentum of $\pi^-$ in the $\NStar$ rest frame,
and $I(\Pars)$ is the normalization integral. 
The fitting parameters, $\Pars$, represent independent helicity or $LS$ couplings,
and masses and widths of resonances (or Flatt{\'e} parameters).

This quasi-log-likelihood sum is scaled by a constant factor,  
$s_W \equiv \sum_i W_i/\sum_i W_i^2$ to account for the effect of the background subtraction on the statistical uncertainty. 
To determine $\Pars$, 
\begin{align}
-2\ln\Like(\Pars)=
& -2 s_W\,\sum_i W_i\ln \PDF_{\rm sig}(m_{p\pi~i},\POmega_i|\Pars)  \notag\\
= & -2 s_W\, \sum_i W_i \ln |\Mat(m_{p\pi~i},\POmega_i|\Pars)|^2
+ 2 s_W\, \ln I(\Pars) \sum_i W_i \notag\\
&~~~~~~~~~ -2 s_W\, \sum_i W_i \ln[ \Phi(m_{p\pi~i})\epsilon(m_{p\pi~i},\POmega_i) ].
\end{align}
is minimized.
The last term does not depend on the fitted parameters $\Pars$ and is therefore dropped.
The efficiency still appears in the normalization integral.  The integration is done without the need to parameterize the efficiency,
by summing the matrix element squared over the simulated events that are
generated uniformly in phase space and pass
through the detector modelling and the data selection procedure,
\begin{eqnarray}
I(\Pars)& \equiv & \int \left|\Mat(m_{p\pi},\POmega|\Pars)\right|^2
\Phi(m_{p\pi})
\epsilon(m_{p\pi},\POmega)
\,dm_{p\pi}\,d\POmega \notag\\
&\propto & \sum\limits_{j}^{N_{\rm MC}} w_j^{\rm MC} \left|\Mat(m_{p\pi~j},\POmega_{j}|\Pars)\right|^2,
\label{SUPPeq:signor}
\end{eqnarray}
where $w_j^{\rm MC}$ are the weights given to the simulated events
to  improve the agreement between data and simulations.

As described above, 
the PID cuts on the hadron tracks, 
$p,\pi$ candidates are not applied in the simulation sample,
instead their effect is considered by the PID weights determined using the \texttt{PIDCalib} package. 
The PID weight is the probability (efficiency) of a track passing a specific PID requirement, 
and is a function of the track momentum, 
pseudo-rapidity and event multiplicity. 
For each $\Lb$ candidate, 
the weight is determined as the product of the two PID weights on $p$ and $\pi$.

%Another weight is introduced to account for the difference of the $\Lb$ kinematic $(\pt,y)$ distribution between data
%and simulation. The inconsistency of the $\Lb$ kinematic distribution depends only on the Monte Carlo generator but not on the $\Lb$ decay modes.
%So the weights that were used in the $\LbJpsipK$ analysis to correct the $\Lb$ $(\pt,y)$ distribution are used again.
%In the end, the kinematic weight is multiplied by the PID weights to determine a total weight for each event.
%The weighted Monte Carlo is used to calculate the selection efficiency \etc in this analysis.


\subsection{{$N^*$ resonances }}

The conventional resonances contributing to the $\Lb$ channel are excited $N^{*}$ states that decay to $p \pi^-$. 
The isospin violated decay $\Lb\to \jpsi \Delta^*$, $\Delta^*\to \proton \pim$ is expected to be suppressed. 
A preliminary study on similar decay gives the isospin violated decay  
to the allowed decay $\Gamma(\Lb \to \jpsi \Sigma)/\Gamma(\Lb \to \jpsi \Lambda)<1\%$ at 95\% confidence level. 
Thus $\Delta^*$ is not considered in this study.

Table~\ref{tab:Lstar} lists total 14 $N^*$ states that we consider. 
They are all established resonances (overall status 3 or 4 stars). 
%All $N^*$ resonances including lower stars are listed in Table~\ref{tab:NstarAll} of Appendix~\ref{sec:app:Nstar}. 
The Breit-Wigner amplitude is used for all the resonances, 
except for the $N(1535)$ state, 
that is described by a Flatt{\'e} function. 
Because the resonance has significant coupling to $\eta N$ channel whose open threshold is very close to the mass of the state. 
In the $\pi N$ elastic scattering, 
the resonance is shown by a sharp cusp~\supercite{Arndt:2005dg, Arndt:2006bf}.  
Since the angular momenta in the two coupling channels are all zero, the Flatt{\'e} function can be simplified as
\begin{equation}
{\rm F}(m | M_0, g_{N\pi}, g_{n\eta} ) = \frac{1}{M_0^2-m^2 - i \left(g^2_{N\pi}\rho_{N\pi}(m)+g^2_{n\eta}\rho_{n\eta}(m)\right)} \,,
\label{SUPeq:breitwigner}
\end{equation}
where $\rho_k(m)=2q_k(m)/m$ is the phase-space factor, $g_{k}$ is the coupling,
and $q_k$ is the momentum of either $\pi$ or $\eta$ in the $\NStar$ rest frame for channel $k=N\pi$ or $n\eta$.
Here $N$ can be either a proton $p$ or a neutron $n$; 
their mass difference is ignored and use the proton mass instead. 
When $m<m_n+m_\eta$, $q_{n\eta}$ becomes imaginary for continuance, \ie
\begin{equation}
q_k(m) = i\frac{\sqrt{|(m^2-(m_n+m_\eta)^2)(m^2-(m_n-m_\eta)^2)|}}{2m},
\end{equation}
where $m_n$ and $m_\eta$ are the masses of $n$ and $\eta$, respectively.
According to Ref~[\cite{Anisovich:2011fc}],  
$g^2_{k}\rho_{k}(M_0)=M_0{\cal B}_{k}\Gamma_0$, 
where $M_0$ and $\Gamma_0$ are the Breit-Wigner mass and width of the resonance respectively, 
while ${\cal B}_k$ is the branching fraction of the resonance decaying to the channel $k$. 
${\cal B}_{n\eta}/{\cal B}_{N\pi}=(0.95\pm0.03)\%$ is used \supercite{PDG}, 
and assume that the two channels saturate the full decay width of the resonance.

An S-wave non-resonant (NR) contribution for $J^P=1/2^-$  $p\pim$ is included. 
The mass dependent term is described by the form $1/m^2$ which is found to be significant improving the fit.

Due to the limited sample size, 
the fit cannot converge including all resonances with all helicity couplings. 
Each $N^*$ resonance is described by maximum 4 (6) independent $LS$ couplings for the spin equal to (great than) 1/2, 
where $L$ is the orbital angular momentum of the $\Lb$ decay and $S$ is the total spin of the $\jpsi N^*$ system. 
High $L$ couplings are expected to be suppressed by the angular momentum barrier, 
so the couplings is selected by first allowing low $L$ contribution. 
Table~\ref{tab:Lstar} lists the models including $N^*$ states with independent $LS$ couplings used. 
The "Nominal Model" (NM) contains 10 $N^*$ resonances and a NR component described by 40 free parameters, 
in which each $L$ component of each resonance that improves the $\twolnL$ great than 9 units. 
``Extended Model" (EM) contains the maximum number free parameters that can achieve a converged fit.  
As the ``Extended" model have too many parameters and hard to obtain accuracy uncertainty, 
it will be used as the systematic study and the significance evaluation.

\begin{table}[htb]
\centering
\caption{The $N^*$ resonances used in the different fits. 
Parameters are taken from the PDG \supercite{PDG}. 
The number of $LS$ couplings is also listed for the ``Nominal Model'' (NM) and ``Extended  Model" (EM).
To fix overall phase and magnitude conventions, 
which otherwise are arbitrary, 
the $N(1535)$ coupling of lowest $LS$ is set to be (1,0).  
A zero entry means the state is excluded from the fit. }
\label{tab:Lstar}
\vspace{0.2cm}
\begin{tabular}{lcccccc}
\hline\\[-2.5ex]
State & $J^P$ & $M_0$ (MeV) & $\Gamma_0$ (MeV)&  NM & EM   \\
\hline \\[-2.5ex]
$N(1440)$ &1/2$^+$ 	& 1430 & 350 	& 4 & 4 \\
$N(1520)$ &3/2$^-$ 	& 1515 & 115 	& 3 & 3 \\
$N(1535)$ &1/2$^-$ 	& 1535 & 150 	& 3 & 3 \\
$N(1650)$ &1/2$^-$ 	& 1655 & 140 	& 4 & 4 \\
$N(1675)$ &5/2$^-$ 	& 1675 & 150 	& 5 & 5\\
$N(1680)$ &5/2$^+$ 	& 1685 & 130 	& 3 & 3 \\
$N(1700)$ &3/2$^-$ 	& 1700& 150 	& 3 & 3 \\
$N(1710)$ &1/2$^+$ 	& 1710 & 100 	& 4 & 4 \\
$N(1720)$ &3/2$^+$ 	& 1720 & 250	& 5 & 5 \\
$N(1875)$ &3/2$^-$ 	& 1875 & 250 	& 3 & 3 \\
$N(1900)$ &3/2$^+$ 	& 1900 & 200 	& 0 & 3 \\
$N(2190)$ &7/2$^-$	& 2190& 500 	& 0 & 3 \\
$N(2300)$ &1/2$^+$ 	& 2250 & 400 	& 0 & 3\\
%$N(2220)$ &9/2$^+$ 	& 2250 & 400 	& 0 & 3\\
%$N(2250)$ &9/2$^-$ 	& 2275 & 500 	& 0 & 3 \\
%$N(2600)$ &11/2$^-$ 	& 2600& 650 	& 0 & 3\\
NR $p\pi$  &1/2$^-$ 	& - & - 		& 4 & 4 \\
%$\ZP(4380)$& 3/2$^-$& 4  & 4 & 4 & 4\\
%$\ZP(4450)$& 5/2$^+$ & 4  & 4 & 4 & 4\\
\hline
\multicolumn{4}{l}{Free parameters}&  82 & 100 \\\hline
%\multicolumn{2}{l}{$\twolnL$} &  $-2164$ & $-2207$ & $-2218$ & $-2230$\\\hline
\end{tabular}
\end{table}

\subsection{{Parameters describing the $P_c^+$ states}}
\label{sec:Pcpar}
Because of limited statistics, 
the strategy is to check whether or not the data is consistent with the $\Lb\to\jpsi p K^-$ decays studied previously~\supercite{LHCb-PAPER-2019-014}, 
thus it is required to use constraints as many as possible from that study.

Each $P_c^+$ state is described by total 10 parameters: 
2 parameters for mass and width, 
6 for independent helicity couplings of the $P_c^+\to \jpsi p$ decay, 
${\cal H}_{\lambda_\psi^{Pc}\,,\lambda_p^{Pc}}^{P_{cj}\to\psi p}$, 
and 2 for the complex $LS$ coupling ratio of the $\Lb\to P_c^+ \pi^-$ decays, 
$B_{L=J_{Pc}+1/2}/B_{L=J_{Pc}-1/2}$. 
To obtain a reliable fit,  
some free parameters are reduced: 
for each $P_c^+$ state, 
the mass and width to the measured values from the $\Lb\to\jpsi p K^-$ decay is fixed~\supercite{LHCb-PAPER-2019-014}.
Besides,
the smallest allowed orbital angular momentum is $L=2$ for $P_c^+$ states.
%as well as the 6 helicity couplings of the $P_c^+$ decays, 
%because the couplings are independent of how the $P_c^+$ state is produced. 
%Since the higher $L=J_{Pc}+1/2$ contribution is suppressed related to the lower $L=J_{Pc}-1/2$, we further fix their ratio to 0, in addition the higher $L$ contribution is found insignificant in the current data. 
Then,
totally 2 free parameters for $P_c^{+}$ states are obtained. 
%2 free parameters accounting for the overall magnitude and phase related to the reference channel $N(1535)$ 
%with the smallest $(L,S)=(0,\frac{1}{2})$ in the $\Lb$ decay, 
%and other 2 free parameters for the ${P_{cj}\to\jpsi p}$ decay.
%and other 2 free parameters for the ratio $B_{L=J_{Pc}+1/2}/B_{L=J_{Pc}-1/2}$. 

In the nominal model,
all the $P_c^+$ states have negative parity according to the prediction of melecular model,
$J^{P}=1/2^{-}$ for $P_{c}(4312)$ and $P_{c}(4400)$,
and $J^{P}=3/2^{-}$ for $P_{c}(4380)$ and $P_{c}(4457)$\supercite{LIU2019237},
this spin-parity assignment is used in this analysis.











