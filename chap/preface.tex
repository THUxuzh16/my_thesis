\chapter{Preface}

The dissertation is written by the author to apply the degree of doctor of philosophy in physics.
The author reviewed recent progresses in hadron spectroscopy,
in the view of experimentical and theoretical respectively,
as in Chapter.~\ref{chap:introduction}.
As the author is a menber of \lhcb collaboration, 
the possibility to study some exotic states at \lhcb was also discussed.

Chapter.~\ref{chap:lhcb} is a brief review to \lhcb spectrometer,
operating at \lhc accelerator.
Every subdetector at LHCb is introduced,
especially for the callorimeter system,
as the author joins the \ecal simulation study for \upgradetwo.
\lhcb is being upgraded now,
some new characters of every subdetector after \upgradeone are also introduced in this Chapter. 


The author within \lhcb collaboration is a principle member in the amplitude analysis of $\Bp\to\jpsi\phi\Kp$ decay with Run 1 and Run 2 data sample,
and participated in every step of this study from decay channel reconstruction to systematic uncertainties estimation cooperated with Prof.\,Liming Zhang and Prof.\,Tomasz Skwarnicki.
Besides, Prof.\,Yuanning Gao, Zhihong Shen and Mengzhen Wang also offered a lot of help in the analysis and review process.
This study is a key component of this dissertation,
which will be discussed in Chapter~\ref{chap:Zcs_study}.
The key conclusions of this analysis have been published in PRL\supercite{} authored by \lhcb collaboration,

The study of $\Lb\to\Lc\Kp\Km\pim$ decay is an early research topic to the author, under the guidance of Prof.\,Yuanning Gao and Prof.\,Liming Zhang.
Every step of this analysis has been performed by the author,
and the details will be introduced in Chapter~\ref{chap:open_pentaquark}. 
The result of this study has been published in PLB\supercite{} authored by \lhcb collaboration.

The author is a member of the \ecal \upgradetwo simulation group,
where he contributed to the parameterized fast simulation and Silicon-Tungsten \ecal full simulation,
guided by Prof.\,Liming Zhang, Prof.\,Yuanning Gao, Prof.\,Patrick Robbe and Adam Davis.
The author achieved neutral particle reconstruction with timing in DELPHES married \gauss framework,
also involved in the development of this framework.
A standalone simulation tool for Silicon-Tungsten \ecal at \lhcb environment is developed by the author.
Besides,
the performance of several decay channels have been studied with this tool,
cooperated with Prof.\,Liming Zhang, Prof.\,Zhenwei Yang, Prof.\,Yuanning Gao, Zirui Wang and Zhihong Shen.
Some results are summarized in Chapter~\ref{chap:ecal} and a \lhcb internal document.

The author have ever been involved directly in the analysis of $\Lb\to\jpsi\proton\Kp$ decay,
who mainly worked in the selection optimation, mass resolution and parts of systematic unceratainties,
cooperated with Prof.\,Liming Zhang and Prof.\,Tomasz Skwarnicki.
A brief summary is demonstrated in Appendix~\ref{app:pentaquark_jpsipk},
this analysis is published in PRL\supercite{}.
As the author didn't follow every analysis step, 
this study is not put in main text of this dissertation.
A similar decay, $\Lb\to\jpsi\proton\pim$, is being studied by the author, 
which is still at initial stage.
The amplitude results are being discussed in collaboration,
and the process of this study is suspended as warting for results from $\Lb\to\jpsi\proton\Kp$ amplitude analysis.
Only selection optimation of this study is shown in Appendix.~\ref{chap:pentaquark_jpsippi}.


%The amplitude analysis of $\Bp\to\jpsi\phi\Kp$ decay with Run 1 and Run 2 data sample has been carried out by the author within \lhcb collaboration,
%cooperated with Prof. Liming Zhang and Prof. Tomasz Skwarnicki,
%also assisted by Prof. Yuanning Gao, Zhihong Shen and Mengzhen Wang.
%The author performed ever step of this study from reconstruction scripts to systematic unceratainties study,






