
\section{Conclusions}

By combining Run 1 and Run 2 data sample, 
and improving the data selection, we gained a factor of 5.5 in the $\Bpdecay$ signal statistics relatively to the previous amplitude analysis of these decays, 
while obtaining 5.8 smaller fractional non-$\Bp$ background at the same time.
The increased statistical power of the data revealed some inadequacies of the amplitude model developed to describe the previous Run 1 data sample. 

The two $1^{++}$ ($\Xone$ and $\Xtwo$) 
and two $0^{++}$ ($\Xthree$ and $\Xfour$) 
states decaying to $\jpsi\phi$ are confirmed together with their quantum number assignments.
The new mass and width determination supersede our previously published values.

An additional $1^{++}$ $X(4685)$ state is observed with relatively narrow width ($\Gamma=126\pm15_{-41}^{+37}$\mev) at more than $15\sigma$.
New partial $\jpsi\phi$ wave $X(4630)$ state is required with at least $6\sigma$ significance with $1^{-+}$ $3\sigma$ better than the $2^{-+}$ hypothesis. 
Other $J^{PC}$ hypotheses are rejected for more than $5\sigma$.

Pattern of masses and quantum numbers is not the one aligning with the expectations for pure charmonium states, 
thus tetraquark $c\bar{c}s\bar{s}$ structures are likely contributing to the mass spectrum.
Direct color binding is more likely to be responsible than  meson-meson interactions, 
because of the rich spectrum of quantum numbers and the decay widths larger than 
for the narrow $c\bar{c}u\bar{d}$ and $c\bar{c}uud$ states observed near the constituent hadron-hadron mass thresholds. 

While broader $c\bar{c}u\bar{d}$ states decaying to $\jpsi\pi^+$ or $\psi'\pi^+$ were observed in $B\to\psi\pi^+K$ decays, 
our data requires for the first time $c\bar{c}u\bar{s}$ states decaying to $\jpsi K^+$. 
A relatively narrow, 
$\Gamma=131\pm15_{-26}^{+26}$ \mev, 
$1^+$ $Z_{cs}(4000)^+\to J/\psi K^+$ state is observed at $4003\pm6_{-14}^{+4}$ \mev, 
with a significance of more than $15\sigma$. 
A broad $1^+$ $Z_{cs}(4220)^+$ state is also needed at $6\sigma$ significance.






