
\section{Efficiency}
\label{sec:efficiency}

The total efficiency for \LbLckkpi and \LbLcDs decays is defined as
\begin{equation}
\epsilon_{\rm tot} =\epsilon_{\rm acc}\cdot\epsilon_{\rm rec\& sel\&hlt |acc} \cdot \epsilon_{\rm L0 trig | sel\&hlt} ,
\end{equation}
where $\epsilon_{\rm acc}$ is the geometrical acceptance of the LHCb detector, 
 $\epsilon_{\rm rec\&sel|acc}$ stands for the reconstruction and selection efficiency, 
 $\epsilon_{\rm trig|sel}$ is the trigger efficiency.
The summary of efficiencies is shown in Table~\ref{tab:efflist}.


\begin{table}[htp]
\centering
\caption{Relative efficiency values. Uncertainties listed are due to finite MC statistic, and are discussed in detail in Sec.~\ref{sec:systematic}}
\vspace{0.2cm}
\label{tab:efflist}
\begin{tabular}{lcc}\hline\hline
	$\frac{\epsilon(\LbLckkpi)}{\epsilon(\LbLcDskkpi)}$ & value \\\hline

Acceptance&\multicolumn{2}{c}{$0.9947 \pm 0.0061 $}\\
Rec. Sel.and Hlt (No PID)&\multicolumn{2}{c}{$0.8286 \pm 0.0128 $}\\
Tracking Correction&\multicolumn{2}{c}{$1.0003 \pm 0.0031 $}\\
PID &\multicolumn{2}{c}{$1.0032 \pm 0.0064 $}\\
L0 Trigger & $ 0.9458 \pm 0.0168$ \\
\hline
Total & $0.7822 \pm 0.0198 $\\
\hline
\end{tabular}
\end{table}


\subsection{Geometrical acceptance}
The geometrical acceptance requires all final state particles in \LbLckkpi(\LbLcDs) decays to be within a polar angle $\theta$ from 10 to 400\mrad.
It is studied with generator level MC events for both \LbLckkpi and \LbLcDs decays respectively, 
generated without the daughter in \lhcb acceptance requirement. 
Then the requirement is applied to the MC samples, 
and the acceptance efficiency is calculated as $\epsilon_{\rm acc}=N_{\rm acc}/N_{\rm GL}$, 
where $N_{\rm acc}$ is the number of events after the \lhcb acceptance cuts and $N_{\rm GL}$ is the number of events 
with only the requirement of the fiducial cuts.


\subsection{Reconstruction, selection and HLT efficiency}
The reconstruction and selection efficiency is defined as $\epsilon_{\rm rec \& sel|acc}=N_{\rm rec \& sel \& hlt}/N_{\rm acc}$, 
where $N_{\rm rec \& sel \&hlt }$ is the signal yield of the full simulation sample after event selection 
and passing HLT1 and 2 requirements  (without PID requirement) and $N_{\rm acc}$ is the corresponding number of generated events in the \lhcb acceptance.
The efficiency is estimated by candidate-by-candidate method using weighted Monta Carlo samples, while HLT and MVA are also included.

\subsection{Tracking efficiency}
The tracking efficiency is estimated from simulation, 
and calibrated with data by introduction factor of the MC efficiencies. 
The correction of MC efficiency is taken as a function of the momentum and pseudorapidity of the track, 
and the event-by-event correction factor is taken as the product of the correction factors of each final tracks. 
After the correction, we calculated the ratio tracking efficiencies with the weight.


\subsection{\lone Trigger efficiency}
\label{sec:triggereff}

The trigger efficiency tables are taken from calibration data, with 2011 and 2012 separately. 
These tables are used to compute L0 Hadron TOS decision, in bins of their real ET when arriving to the HCAL surface, 
and separately for the inner and outer regions of the calorimeter. 
We reweight our MC sample using these tables and the trigger efficiency on event is computed 
by combining the efficiencies corresponding to each one of the final tracks, 
assuming \Lb fires the L0 trigger if at least one track triggers it.
Details see the
\href{https://twiki.cern.ch/twiki/bin/viewauth/LHCbPhysics/CalorimeterObjectsToolsGroupDOC#L0_Hadron_trigger_efficiencies}{link}.


\subsection{PID selection efficiency}
\label{sec:getPIDeff}
It is known that the PID variables are not well described by the simulation, 
therefore a data-driven method is used instead. 
The PID efficiencies for the $\proton, \antiproton, \pi^\pm, K^\pm$ are applied using the \texttt{PIDCALIB} package, 
which gives 3-D histograms that shows the efficiency of a certain PID requirement of a track as a function of the momentum and  pseudorapidity of the track, 
and the multiplicity of the event. 
The single event efficiency are taken from this method, 
and the total PID efficiency of \LbLckkpi(\LbLcDs) is derived by averaging the single event efficiencies of the candidates 
in the full simulated signal samples that pass all the event selections (except for PID cut).

For the calibration of proton PID efficiencies, 
we use the calibration samples from $\Lc$ decays to cover the similar kinematic space as the protons from our decay modes.





